\chapter{Abstract}
The subject of this thesis is process scheduling in the Linux kernel, with a special focus on the Completely Fair Scheduler. A particular emphasis is placed on how groups of tasks are scheduled. And how the user can interact with them and control their resources. Chapter \ref{ch:introduction} analyzes the basics of the Linux kernel. It introduces the user to the design of the kernel and defines the notions useful for understanding the scheduling process. Chapter \ref{ch:sched} then describes the scheduler from a theoretical point of view, comparing it to previous implementations. And detailing how it handles single tasks. Chapter \ref{chap:implementation} covers the implementation of the scheduler. It analyzes some of the core functions involved in the scheduling process. Finally, chapter \ref{ch:cgroup} describes how the scheduler treats groups of processes, and what are the advantages of scheduling tasks as a group instead of singularly. This chapter also covers the basic working of \verb|cgroups|, and how they can be used to control the CPU resource for a group of tasks. This work is based on version 4.20.13 of the Linux kernel.
